\title{Probabilistic retrievals of precipitation for the GPM constellation using machine learning}
\author{Teodor Norrestad, Simon Pfreundschuh, Patrick Eriksson}
\date{\today}

\documentclass[12pt]{scrartcl}

\usepackage{todonotes}

\begin{document}
\maketitle

\section{Introduction}

\todo{Significance of rain}

It is difficult to understate the importance of precipitation within  the
Earth's climate system or its relevance for many areas of human activity.

\todo{Evolution of GPROF}

The initial version of the Goddard Profiling Algorithm (GPROF) was designed to
provide a computationally efficient method for retrieving precipitation profiles
and surface precipitation from the passive observations from the Tropical
Rainfall Measurement Mission (TRMM, \citet{grecu04}) Microwave Imager. Since
then, it has gone through several stages of development \citep{kummerow01,
kummerow11} and has evolved into a fully-parametric scheme that is used to retrieve
precipitation from all sensors of the Global Precipitation Measurement Mission
(GPM, \citet{hou14}).

\citet{kummerow96}

\todo{Differences between QRNN and MCI}

\todo{What we do in this article}

\end{itemize}

\section{Methods and data}

\subsection{Monte carlo integration}

\subsection{Quantile regression neural networks}

\subsection{Estimating the posterior distribution}

\subsection{Datasets}

\section{Results}

\subsection{Internal validation}

\subsection{External validation}

\section{Discussion}

\section{Conclusions}


\end{document}
